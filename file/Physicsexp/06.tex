\documentclass[10pt,a4paper]{article}	%字体、纸张
\usepackage{xeCJK}	%中文
\usepackage[a4paper,left=20mm,right=20mm,top=25mm,bottom=20mm]{geometry}	%页边距
\usepackage{fancyhdr}	%页眉、页脚
\usepackage{indentfirst}	%首行缩进
\usepackage{graphicx}	%图片
\usepackage{textcomp}
\usepackage{subfigure}	
\usepackage{enumitem}
\usepackage{tabularx}
\usepackage{multirow}
\usepackage{caption}
\usepackage{amsmath}	%公式对齐
\usepackage{tikz-feynman}

\newcommand{\nexp}{透镜参数的测量}

%————页眉、页脚设置————
\thispagestyle{plain}
\pagestyle{fancy}
\fancyhf{}
\fancyhead[R]{PB22000195 王元叙}
\fancyhead[L]{\nexp}	%————————————
\fancyfoot[C]{\thepage}
\renewcommand{\headrulewidth}{0pt}
\renewcommand{\footrulewidth}{0pt}
%————————————

\setenumerate[1]{itemsep=0pt,partopsep=0pt,parsep=\parskip,topsep=5pt}
\setlength{\parindent}{2em}
\renewcommand\arraystretch{1.4}

\makeatletter
\newenvironment{figurehere}
{\def\@captype{figure}}
{}
\newenvironment{tablehere}
{\def\@captype{table}}
{}
\makeatother

\begin{document}
	%————起始————
	\vspace*{-5em}
	\begin{center}
		\includegraphics[width=0.6\textwidth]{Picture//USTC}\\
		\Large \textbf{大学物理-基础实验|实验报告}\\[5mm]

		\normalsize
		\begin{tabular}{ll}
			姓名 & \textbf{王元叙}\\
			学号 & \textbf{PB22000195}\\
			班级 & \textbf{22级少年班学院5班}\\
			日期 & \textbf{2023年5月22日}\\	
		\end{tabular}\\[5mm]

		\LARGE \textbf{\nexp}\\[5mm]	

	\end{center}
	%————————————

	%————正文————
	\section{实验目的}

	\begin{enumerate}
		\item 掌握光源、物、像间的关系以及球差、色差产生的原因;
		\item 熟练掌握光具座上各种光学元件的调节并且测量薄透镜的焦距。
	\end{enumerate}


	\section{实验装置}
	光具座、白炽灯光源、透镜架、“1”字屏、毛玻璃、像屏、一个未知焦距凸透镜、一个焦距大致为15mm凸透镜,一个未知焦距凹透镜、钢卷尺、铅笔、平面镜

	\section{实验原理}
	\subsection{高斯成像公式}
	在近轴条件下高斯公式成立, 设 $p$ 为物距, $p^{\prime}$ 为像距, 物方焦距(也称前焦距)为 $f$, 像方焦 距 (也称后焦距) 为 $f$ 则有:
	\[
		\frac{f^{\prime}}{p^{\prime}}+\frac{f}{p}=1 \tag{1}
	\]
	由于在空气中 $f=-f^{\prime}$, 高斯公式变成
	\[
		\frac{1}{p^{\prime}}-\frac{1}{p}=\frac{1}{f^{\prime}} \tag{2}
	\]
	\subsection{自准直法}
		位于焦点F上的物A所发出的光经过透镜变成平行光。再经平面镜M反射后可在物屏上得到清晰的倒立像A'。如图1:
		
		\begin{figurehere}
			\centering
			\includegraphics[width=0.55\linewidth]{pics/1.png}
			\caption*{\bf 图1: 自准直法测量焦距原理图}
		\end{figurehere}
	\subsection{物像距法(公式法)}
		固定透镜,将物放在距透镜一倍以上焦距处,在透镜的像方某处会获得一清晰的像,物距、像距分别为自透镜中心处至物、像间的距离。
		
		也可以采用如下公式计算焦距
		\[
		\frac{1}{p^{\prime}}+\frac{1}{p}=\frac{1}{f^{\prime}} \tag{3}
		\]
		
		式中不再区分前后焦距,采用以下原则来确定符号:对物距和像距,实物与实像时取正号,虚物和虚像时取负号;对透镜焦距f,凸透镜取正号,凹透镜取负号。
	\subsection{位移法}
		物距在一倍焦距和二倍焦距之间时,在像方可以获得一放大的实像,物距大于二倍焦距时,可以得到一缩小的实像。当物和屏之间的距离$L$大于$4f$时,固定物和屏,移动透镜至两特定位置C、D处,在像屏上可分别获得放大和缩小的实像。C、D间距离为$l$,通过物像公式,可得
		\[
			f=\frac{L^2-l^2}{4 L} \tag{4}
		\]
	\subsection{辅助透镜法测凹透镜焦距}
		凹透镜是一发散透镜,物经其仅能成虚像,虚像不能用像屏接受,这样无法直接用物成像的方法来计算焦距,但可利用凸透镜成的像作为凹透镜的物,再产生一个实像。注意凹透镜的像方焦点在物空间,物方焦点在像空间。实验中应使凹透镜成像的物距、像距均大于0,才能用屏接收到实像,如图2所示。\\

		\begin{figurehere}
			\centering
			\includegraphics[width=0.55\linewidth]{pics/2.png}
			\caption*{\bf 图2: 辅助透镜法测凹透镜焦距原理图}
		\end{figurehere}

	\subsection{自准直法测凹透镜焦距}
		无法直接使用自准直法测量凸透镜焦距,因此可以使用辅助透镜方法,实验设计图参见附录中原始数据。

	\section{实验步骤}

    \begin{enumerate}
		\item 目测调节。即将所用的元件靠拢,使其光心等高,光轴平行于光学平台。
		\item 利用成像原理细调元件共轴。
		\item 物像距法测量凸透镜焦距:物距固定不变单次测量,像距6次测量取平均。
		\item 位移法测量凸透镜焦距数据:物像距离固定不变单次测量,透镜位移量6次测量取平均。
		\item 自准直法测量凸透镜焦距:直接测得焦距6次,测量前三次后翻转透镜测量后三次。
		\item 物像距法测量凹透镜焦距:物距固定不变单次测量,像距3次测量取平均。
		\item 自准直法测量凹透镜焦距:直接测得焦距3次。
        \item 整理仪器,结束实验。
    \end{enumerate}

	\section{实验数据与分析}

	\subsection{物像距法测量凸透镜焦距}

	\begin{tablehere}
		\caption*{\bf 表1 物像距法测量凸透镜焦距数据}
		\noindent	
		\begin{center}
			\newcolumntype{Y}{>{\centering\arraybackslash}X}
			\begin{tabularx}{0.85\textwidth}{|l|Y|Y|Y|Y|Y|Y|}
				\hline
				物距/$\mathrm{mm}$ & \multicolumn{6}{c|}{144.3} \\ \hline
				像距/$\mathrm{mm}$ & 326.7 & 324.3 & 325.1 & 327.2 & 324.9 & 321.1 \\ \hline
				平均物距/$\mathrm{mm}$ & \multicolumn{6}{c|}{324.9} \\ \hline
				焦距/$\mathrm{mm}$ & \multicolumn{6}{c|}{99.9} \\ \hline
			\end{tabularx}
			\vspace*{1em}
		\end{center}
	\end{tablehere}

	{\bf 不确定度分析过程:}

	像距 $p^{\prime}$ 平均值
	$$
	\overline{p^{\prime}}=\frac{1}{n} \sum_{i=1}^n p_i^\prime=\frac{326.7 + 324.3 + 325.1 + 327.2 + 324.9 + 321.1}{6} \mathrm{~mm}=324.9 \mathrm{~mm}
	$$

	由高斯公式 (2) 求得焦距
	$$
	f=\frac{p p^{\prime}}{p+p^{\prime}}=\frac{324.9 \times 144.3}{324.9+144.3} \mathrm{~mm}=99.9 \mathrm{~mm}
	$$

	像距的 A 类不确定度
	$$
	u_A=\sqrt{\frac{1}{(n-1) } \sum_{i=1}^n\left(p_i^{\prime}-\overline{p^{\prime}}\right)^2} \mathrm{~mm}=2.54\mathrm{~mm}
	$$

	像距的 B 类不确定度
	$$
	\Delta_{B, p^{\prime}}=\sqrt{\Delta_{\text {仪 }}^2+\Delta_{\text {估 }}^2}=\sqrt{2^2+0.5^2} \mathrm{~mm}=2.06 \mathrm{~mm}
	$$

	像距的展伸不确定度
	$$
	\begin{aligned}
	U_{p^{\prime}, P} & =\sqrt{\left(t_P \frac{u_A}{\sqrt 3}\right)^2+\left(k_P \frac{\Delta_{B, p^{\prime}}}{C}\right)^2} \\
	& =\sqrt{(2.57 \times \frac{2.54}{\sqrt 3})^2+\left(1.96 \times \frac{2.06}{3}\right)^2} \mathrm{~mm} \\
	& =3.94 \mathrm{~mm}, P=0.95
	\end{aligned}
	$$

	物距 $p$ 展伸不确定度
	$$
	U_{p, P}=k_P \frac{\Delta_{B, p^{\prime}}}{C}=1.35 \mathrm{~mm}
	$$

	焦距 $f$ 的展伸不确定度
	$$
	\begin{aligned}
	U_{f, P} & =\sqrt{\left(\frac{\partial f}{\partial p} U_{p, P}\right)^2+\left(\frac{\partial f}{\partial p^{\prime}} U_{p^{\prime}, P}\right)^2} \\
	& =\sqrt{\left(\frac{p^2}{\left(p+p^{\prime}\right)^2} U_{p, P}\right)^2+\left(\frac{p^{\prime 2}}{\left(p+p^{\prime}\right)^2} U_{p^{\prime}, P}\right)^2} \\
	& = 1.9 \mathrm{~mm}, P=0.95
	\end{aligned}
	$$

	焦距 $f$ 最终结果
	$$
	g=(99.9 \pm 1.9) \mathrm{mm} \quad(P=0.95)
	$$



	\subsection{位移法测量凸透镜焦距}

	\begin{tablehere}
		\caption*{\bf 表2 位移法测量凸透镜焦距数据}
		\noindent
		\begin{center}
			\newcolumntype{Y}{>{\centering\arraybackslash}X}
			\begin{tabularx}{0.85\textwidth}{|l|Y|Y|Y|Y|Y|Y|}
				\hline
				物像距/$\mathrm{mm}$ & \multicolumn{6}{c|}{540.3}  \\ \hline
				位移/$\mathrm{mm}$ & 284.2 & 281.9 & 283.0 & 283.2 & 282.7 & 282.1 \\ \hline
				平均位移/$\mathrm{mm}$ & \multicolumn{6}{c|}{282.9} \\ \hline  	
				焦距/$\mathrm{mm}$ & \multicolumn{6}{c|}{98.64} \\ \hline  	
			\end{tabularx}
			\vspace*{1em}
		\end{center}
	\end{tablehere}

	\subsection{自准直法测量凸透镜焦距}

	\begin{tablehere}
		\caption*{\bf 表3 自准直法测量凸透镜焦距数据}
		\noindent
		\begin{center}
			\newcolumntype{Y}{>{\centering\arraybackslash}X}
			\begin{tabularx}{0.85\textwidth}{|l|Y|Y|Y|Y|Y|Y|}
				\hline
				焦距/$\mathrm{mm}$ & 100.9 & 101.5 & 101.1 & 103.5 & 102.1 & 101.9 \\ \hline
				平均焦距/$\mathrm{mm}$ & \multicolumn{6}{c|}{101.8} \\ \hline  		
			\end{tabularx}
			\vspace*{1em}
		\end{center}
	\end{tablehere}

	\subsection{物像距法测量凹透镜焦距}

	\begin{tablehere}
		\caption*{\bf 表4 物像距法测量凹透镜焦距数据}
		\noindent	
		\begin{center}
			\newcolumntype{Y}{>{\centering\arraybackslash}X}
			\begin{tabularx}{0.85\textwidth}{|l|Y|Y|Y|}
				\hline
				物距/$\mathrm{mm}$ & \multicolumn{3}{c|}{100.9} \\ \hline
				像距/$\mathrm{mm}$ & 203.2 & 203.9 & 203.1 \\ \hline
				平均物距/$\mathrm{mm}$ & \multicolumn{3}{c|}{203.4} \\ \hline
				焦距/$\mathrm{mm}$ & \multicolumn{3}{c|}{200.4} \\ \hline
			\end{tabularx}
			\vspace*{1em}
		\end{center}
	\end{tablehere}

	\subsection{自准直距法测量凹透镜焦距}

	\begin{tablehere}
		\caption*{\bf 表5 自准直法测量凹透镜焦距数据}
		\noindent
		\begin{center}
			\newcolumntype{Y}{>{\centering\arraybackslash}X}
			\begin{tabularx}{0.85\textwidth}{|l|Y|Y|Y|}
				\hline
				焦距/$\mathrm{mm}$ & 204.4 & 206.0 & 206.0 \\ \hline
				平均焦距/$\mathrm{mm}$ & \multicolumn{3}{c|}{205.3} \\ \hline  		
			\end{tabularx}
			\vspace*{1em}
		\end{center}
	\end{tablehere}

	\subsection{利用视差测量透镜焦距}

	使用视差法测量了了凸透镜的焦距。测量得到物距为$215.5\mathrm{~mm}$,像距为 $207.5\mathrm{~mm}$ 计算得到焦距为 $105.7\mathrm{~mm}$

	\section{误差分析}

	本实验中实验误差主要来自于几个方面

	\begin{enumerate}
		\item 钢卷尺测量误差
		\item 透镜安装时偏心导致的对位移法、自准直法测量凸透镜焦距的误差。
		\item 使用肉眼判断成像最清晰位置不准确导致的误差
		\item 凹透镜焦距的测量显著更大,这是由于实验使用了辅助透镜法,而增加透镜后上述误差被显著放大。
	\end{enumerate}

	\section{思考题}

	\textbf {问题 1}\textsl{ 如果在“1”字屏后不加毛玻璃,对实验会有什么影响?}

	毛玻璃可以将光线散射,使得光纤扩散均匀,形成大小一定的光斑。否则由于透镜本身存在的光学缺陷,例如球差、色差、散光等因素导致的光斑扩散情况,导致图像质量下降图像不清晰,影响对像距的测量。毛玻璃可以使得测量更加精确。

	\textbf {问题 2}\textsl{ 自准直法测凸透镜焦距时,如果透镜安装在光具座上时沿光轴方向与光具座中心不重合(偏心),而我们测量距离时测量的是光具座之间的距离(默认为光学元件位于光具座中心位置),这对测量有什么影响?如何消除这一影响?}

	在使用自准直方法进行测量时,直接测量的是光具座之间的距离而不是光学元件之间的距离。因此如果透镜本身存在偏心,那么测量的结果就会有显著的偏差。假设这个偏心的位移量为 $x$ 那么实际上直接测量的值为 $l + x$ ,为了消除这个误差,可以在测量完前三次结果后将透镜反转,是的后三次被测量值为 $l - x$ 通过多次测量取平均值可以较好地将偏心导致的误差降低。

	\textbf {问题 3}\textsl{ 在利用公式法和位移法测凸透镜焦距时,如果透镜安装时也存在上述偏心,对实验测量结果是否有影响?}

	使用公式法测量焦距时,同样假设透镜的偏心位移量为 $x$ ,那么实际测量的物距和像距分别为 $p - x$ 和 $p^\prime + x$ ,那么计算得到的焦距为 $f^\ast = \frac{pp^\prime + (p-p^\prime)x - x ^ 2}{p + p ^ \prime}$ ,差值为 $\Delta f = \frac{(p-p^\prime - x)x}{p + p ^ \prime}$ 。因此此事透镜安装的偏心对实验测量结果有影响,但是影响较小,尤其当物距像距接近时这个影响量可以忽略不计。

	使用位移法对焦距进行测量时则没有如上问题,因为两次测量都受到透镜安装偏心的影响且影响量相同,因此对实验测量结果并没有影响。

\end{document}
