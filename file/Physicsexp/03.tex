\documentclass[10pt,a4paper]{article}	%字体、纸张
\usepackage{xeCJK}	%中文
\usepackage[a4paper,left=20mm,right=20mm,top=25mm,bottom=20mm]{geometry}	%页边距
\usepackage{fancyhdr}	%页眉、页脚
\usepackage{indentfirst}	%首行缩进
\usepackage{graphicx}	%图片
\usepackage{subfigure}	
\usepackage{enumitem}
\usepackage{tabularx}
\usepackage{multirow}
\usepackage{caption}
\usepackage{amsmath}	%公式对齐

\newcommand{\nexp}{声速测量}

%————页眉、页脚设置————
\thispagestyle{plain}
\pagestyle{fancy}
\fancyhf{}
\fancyhead[R]{PB22000195 王元叙}
\fancyhead[L]{\nexp}	%————————————
\fancyfoot[C]{\thepage}
\renewcommand{\headrulewidth}{0pt}
\renewcommand{\footrulewidth}{0pt}
%————————————

\setenumerate[1]{itemsep=0pt,partopsep=0pt,parsep=\parskip,topsep=5pt}
\setlength{\parindent}{2em}
\renewcommand\arraystretch{1.3}

\makeatletter
\newenvironment{figurehere}
{\def\@captype{figure}}
{}
\newenvironment{tablehere}
{\def\@captype{table}}
{}
\makeatother

\begin{document}
	%————起始————
	\vspace*{-5em}
	\begin{center}
		\includegraphics[width=0.6\textwidth]{Picture//USTC}\\
		\Large \textbf{大学物理-基础实验|实验报告}\\[5mm]

		\normalsize
		\begin{tabular}{ll}
			姓名 & \textbf{王元叙}\\
			学号 & \textbf{PB22000195}\\
			班级 & \textbf{22级少年班学院5班}\\
			日期 & \textbf{2023年4月5日}\\	
		\end{tabular}\\[5mm]

		\LARGE \textbf{\nexp}\\[5mm]	

	\end{center}
	%————————————

	%————正文————
	\section{实验目的}

	\begin{enumerate}
		\item[1)] 使用压电陶瓷超声换能器来测定超声波在气体、液体和固体中的传播速度;
		\item[2)] 理解并学会运用压电效应(正逆效应);
		\item[3)] 理解驻波的性质,并正确运用其进行声速测量;
		\item[4)] 理解利用利萨如图的性质进行声速测量的原理。
	
	\end{enumerate}

	\section{实验仪器}
    SV5 型声速测量仪(如图所示,主要部件包括信号源和声速测试仪(含水槽))、双踪示波器、不同长度的有机玻璃棒、不同长度的黄铜棒、游标卡尺等
	
	\begin{figurehere}
		\centering
		\includegraphics[width=0.55\linewidth]{pics/equip.png}
		\caption*{\bf 图1: SV5型声速测量仪}
	\end{figurehere}

	\section{实验原理}
	\subsection{驻波法测量}
	声波在介质(空气、液体)中传播,带动介质发生振动;当振动区间的长度是半波长的整数倍时,可
	以形成稳定的驻波;压电传感器可以测量某个场点的振幅,当压电传感器处在驻波波腹位置时,可测得振
	幅极大值;由此可以测出半波长 $\lambda / 2$

	信号源同时给出信号的频率 $f$,利用波动关系 $v = \lambda f$ 可计算声速大小。
	\subsection{相位比较法测量}
	当信号发生器的信号经过介质传播被重新接收后,与直接输出的信号间会产生相位差 $\Delta \varphi$。将两信号
	分别接入示波器的两个输入通道 CH1 和 CH2,在 CH2 上可观测到利萨如图形。当利萨如图从直线演化
	为斜率相反的直线时,信号接收器即经过了半个波长的距离。由此可测出半波长 $\lambda / 2$;

	信号源同时给出信号的频率 $f$,利用波动关系 $v = \lambda f$ 可计`算声速大小。
	\subsection{时差法测量}
	声速测量仪允许加载固体介质。信号源发出脉冲波,测量仪会给出脉冲波通过介质所需的时间 $\Delta t$;利
	用游标卡尺可测出固体介质(固体棒)的长度 L,利用速度的定义式 $v = {L}/{\Delta t}$ 可计算固体介质中声速大
	小
	\section{实验步骤}
	\begin{enumerate}
		\item 正确连接示波器到声速测量仪和示波器。固定传感器探头位置,调节信号源频率直至示波器上观察测到振幅达到最大值。记录下此时信号源的频率 f 作为谐振频率。
		\item 记录实验室空气温度。
		\item 维持谐振频率,首先连接驻波法测量的电路。
		\item 单方向转动鼓轮移动可动探头;观察示波器波形变化,记录下振幅达到极值时声速测量仪上游标对应的位置。共记录 12 组。

		\item 将水槽中盛水至液面线,连接相位比较法测量的电路。
		\item 单方向转动鼓轮移动可动探头;观察示波器上利萨如图图形变化,记录下利萨如图呈现直线时声速测量仪上游标对应的位置。共记录 8 组。
		
		\item 分别将信号源模式改为金属和非金属,发出脉冲波;连接时差法测量的电路。
		\item 分别测量不同长度黄铜棒和有机玻璃棒长度,读取时间。
		\item 整理器材,结束实验,对以上测量的数据进行处理和误差分析。
	\end{enumerate}


	\section{实验数据}
	在以下实验中,确定正弦波信号源频率为 $f=36853\mathrm{~Hz}$,空气温度为 $t=27^\circ \mathrm{C}$
	在给定温度下计算空气中理论声速:
	$$
	v_s=v_0 \sqrt{1+\frac{t}{273.15}}=347.45 \mathrm{~m} / \mathrm{s}
	$$
	\subsection{驻波法测量空气中声速 \& 相位比较法测量液体中声速}
	
	\begin{center}\begin{minipage}[t]{0.9\textwidth}
	\begin{minipage}[t]{0.48\textwidth}
	\begin{tablehere}
		\caption*{\bf 表1  空气中声速测量(驻波法)实验数据}
		\vspace*{-1em}
		\noindent
		\begin{center}
			\newcolumntype{Y}{>{\centering\arraybackslash}X}
			\begin{tabularx}{0.95\linewidth}{|Y|Y|}
				\hline
				测量次数 & 读数/cm \\ \hline
				1    & 1.190 \\ \hline
				2    & 1.668 \\ \hline
				3    & 2.146 \\ \hline
				4    & 2.598 \\ \hline
				5    & 3.086 \\ \hline
				6    & 3.572 \\ \hline
				7    & 4.042 \\ \hline
				8    & 4.530 \\ \hline
				9    & 4.992 \\ \hline
				10   & 5.470 \\ \hline
				11   & 5.948 \\ \hline
				12   & 6.412 \\ \hline
			\end{tabularx}
		\end{center}
	\end{tablehere}
	\end{minipage}
	\begin{minipage}[t]{0.48\textwidth}
	\begin{tablehere}
		\caption*{\bf 表2  水中声速测量(相位比较法)实验数据}
		\vspace*{-1em}
		\noindent
		\begin{center}
			\newcolumntype{Y}{>{\centering\arraybackslash}X}
			\begin{tabularx}{0.95\linewidth}{|Y|Y|}
				\hline
				测量次数 & 读数/cm  \\ \hline
				1    & 5.194  \\ \hline
				2    & 7.364  \\ \hline
				3    & 9.462  \\ \hline
				4    & 11.364 \\ \hline
				5    & 13.148 \\ \hline
				6    & 15.208 \\ \hline
				7    & 17.444 \\ \hline
				8    & 19464  \\ \hline
			\end{tabularx}
		\end{center}
	\end{tablehere}
	\end{minipage}
	\end{minipage}\end{center}

	\subsection{时差法测量固体中声速}


	
	\begin{center}\begin{minipage}[t]{0.9\textwidth}
		\begin{minipage}[t]{0.48\textwidth}
		\begin{tablehere}
			\caption*{\bf 表3  黄铜棒中测量的数据}
			\vspace*{-1em}
			\noindent
			\begin{center}
				\newcolumntype{Y}{>{\centering\arraybackslash}X}
				\begin{tabularx}{0.95\linewidth}{|Y|Y|}
					\hline
					长度/cm  & 用时/$\mu$s \\ \hline
					17.948 & 57        \\ \hline
					25.796 & 80        \\ \hline
				\end{tabularx}
			\end{center}
		\end{tablehere}
		\end{minipage}
		\begin{minipage}[t]{0.48\textwidth}
		\begin{tablehere}
			\caption*{\bf 表4  有机玻璃棒中测量的数据}
			\vspace*{-1em}
			\noindent
			\begin{center}
				\newcolumntype{Y}{>{\centering\arraybackslash}X}
				\begin{tabularx}{0.95\linewidth}{|Y|Y|}
					\hline
					长度/cm  & 用时/$\mu$s \\ \hline
					22.802 & 133        \\ \hline
					26.854 & 150        \\ \hline
				\end{tabularx}
			\end{center}
		\end{tablehere}
		\end{minipage}
		\end{minipage}\end{center}

	\section{实验结果分析}
	
	\subsection{驻波法测量空气中声速}

	斜率
	$$
	m=0.47553\,\mathrm{cm}
	$$

	截距
	$$
	b=0.71355\,\mathrm{cm}
	$$

	线性拟合的相关系数
	$$
	r=\frac{\overline{nL}-\overline{n}\cdot\overline{L}}{\sqrt{\left(\overline{n^2}-\overline{n}^2\right)\left(\overline{L^2}-\overline{L}^2\right)}}=0.99999008
	$$

	斜率标准差
	$$
	s_m=\lvert m\rvert\cdot\sqrt{\left(\frac{1}{r^2}-1\right)/(n-2)}=0.00066983\,\mathrm{cm}
	$$

	截距标准差
	$$
	s_b=s_m\cdot\sqrt{\overline{n^2}}=0.0049298\,\mathrm{cm}
	$$
	
	波长 $\lambda$ 
	$$
	\lambda =2 m=2\times 0.47553\,\mathrm{cm}=0.95106\,\mathrm{cm}
	$$

	波长 $\lambda$ 的延伸不确定度
	$$
	\begin{aligned}
	U_{\lambda ,P}&=\sqrt{\left(\frac{\partial \lambda }{\partial m}U_{m,P}\right)^2}\\
	&=\sqrt{\left(2U_{m,P}\right)^2}\\
	&=\sqrt{\left(2\times 0.00066983\right)^2}\,\mathrm{cm}\\
	&=1.3397 \times 10^{-3}\,\mathrm{cm},P=0.95
	\end{aligned}
	$$
	
	谐振频率的不确定度
	$$
	\Delta_{B,f}=0.001\,\mathrm{Hz}
	$$
	
	声速 $v$
	$$
	v=f \lambda =350.5\,\mathrm{m/s}
	$$

	声速$v$的延伸不确定度
	$$
	\begin{aligned}
	U_{v,P}&=\sqrt{\left(\frac{\partial v}{\partial \lambda }U_{\lambda ,P}\right)^2+\left(\frac{\partial v}{\partial f}U_{f,P}\right)^2}\\
	&=\sqrt{\left(fU_{\lambda ,P}\right)^2+\left(\lambda U_{f,P}\right)^2}\\
	&=0.49372\,\mathrm{m/s},P=0.95
	\end{aligned}
	$$

	声速 $v$ 最终结果
	$$
	v=\left(350.5 \pm 0.5\right)\,\mathrm{m/s}
	$$
	
	实验温度下声速的理论值 $v_t$
	$$
	v_{t}=v_{0} \sqrt{\frac{t}{t_{0}} + 1}=331.45 \sqrt{\frac{27}{273.15}+1}\,\mathrm{m/s}=347.45\,\mathrm{m/s}
	$$

	相对误差δ=0.88052\%
	$$
	\delta =\frac{\left|{v - v_{t}}\right|}{v_{t}}=\frac{\left|{\left(350.5-347.45\right)}\right|}{347.45}\,\mathrm{}=8.8052 \times 10^{-3}\,\mathrm{\%}
	$$
	

	\begin{figurehere}
		\vspace*{0.3em}
		\centering
		\includegraphics[width=0.6\linewidth]{pics/figure1.jpg}
		\caption*{\bf 图2.1 气体中声速测量的最小二乘拟合图}
		\vspace*{0.8em}
	\end{figurehere}
	

	\subsection{相位比较法测量液体中声速}

	直接读出斜率 $m=-2.016 \mathrm{~cm}$ 波长即为
	$$
	\bar{\lambda}=2|m|=4.0312 \mathrm{~cm} P=0.95
	$$

	于是声速测量结果即为 $v=\lambda f=1485.7 \mathrm{~m} / \mathrm{s}$

	\begin{figurehere}
		\vspace*{0.3em}
		\centering
		\includegraphics[width=0.6\linewidth]{pics/figure2.jpg}
		\caption*{\bf 图2.2 水中声速测量的作图法拟合}
		\vspace*{0.8em}
	\end{figurehere}
	
	\subsection{时差法测量固体中声速}

	金属棒,斜率 m = 0.3396cm/µs,于是得金属中声速为 v = 3396m/s

	有机玻璃棒,斜率 m = 0.2383cm/µs,于是得有机玻璃中声速为 v = 2383m/s

	\section{思考题}
	\subsection{定性分析共振法测量时,声压振幅极大值随距离变长而减小的原因。}

	声波在传播过程中的减弱现象与传播距离、声波频率和界面等因素有关。声波在介质(如空气,液体,
固体)中传播时会因为被介质吸收、横向扩散而产生能量损耗;

	实验中所用声波频率很高,力学知识表明,高频声波更容易被介质影响,其能量耗散也较大,因此在
实验过程中即可观察到明显的振幅变化。

	\subsection{声速测量中驻波法、相位法、时差法有何异同?	}

	相同点:都利用压电陶瓷的正/逆压电效应进行实验\\
	
	不同点:
	\begin{itemize}
		\item[-]  原理不同:驻波法利用驻波测量半波长,相位法利用相位差的两个波形成利萨如图测量半波长。这两个实验中均使用波长计算声速,然而测量固体中声速的实验中使用长度除以时间的公式直接计算。
		\item[-]  波源不同:驻波法和相位法都依靠谐振频率 $f$ 下的连续正弦,而时差法不需要调整频率,利用脉冲波;
		\item[-]  测量标定方式不同:驻波法是通过观察声压振幅达到最大值,相位法是通过观察李萨如图形的周期
		性变化,时差法是直接观察信号发生器上的时间显示。
	\end{itemize}
	\subsection{各种气体中的声速是否相同,为什么?	}
	
	根据气体声速经验公式:
	$$
	v=\sqrt{\frac{\gamma R T_0}{M}} \sqrt{\left(1+\frac{t}{273.15}\right)\left(1+\frac{0.3192 p_w}{p}\right)} \mathrm{m} / \mathrm{s}
	$$

	这表明气体声速受到气体绝热指数和摩尔质量以及湿度的影响,因此在同温同压情况下,不同气体声速也不相同
	%————————————
	\clearpage
	\section*{附录: 原始数据}
	\begin{figurehere}
		\centering
		\includegraphics[width=\linewidth, height=0.9\textheight]{pics/origin.pdf}
	\end{figurehere}
\end{document}
