\documentclass[10pt,a4paper]{article}
\usepackage{xeCJK}
\usepackage[a4paper,left=20mm,right=20mm,top=25mm,bottom=20mm]{geometry}	%页边距
\usepackage{fancyhdr}	%页眉、页脚
\usepackage{indentfirst}	%首行缩进
\usepackage{graphicx}	%图片
\usepackage{textcomp}
\usepackage{subfigure}	
\usepackage{enumitem}
\usepackage{tabularx}
\usepackage{multirow}
\usepackage{caption}
\usepackage{amsmath}	%公式对齐
\usepackage{tikz-feynman}

\newcommand{\nexp}{医学物理实验}

%————页眉、页脚设置————
\thispagestyle{plain}
\pagestyle{fancy}
\fancyhf{}
\fancyhead[R]{PB22000195 王元叙}
\fancyhead[L]{\nexp}	%————————————
\fancyfoot[C]{\thepage}
\renewcommand{\headrulewidth}{0pt}
\renewcommand{\footrulewidth}{0pt}
%————————————

\setenumerate[1]{itemsep=0pt,partopsep=0pt,parsep=\parskip,topsep=5pt}
\setlength{\parindent}{2em}
\renewcommand\arraystretch{1.4}

\makeatletter
\newenvironment{figurehere}
{\def\@captype{figure}}
{}
\newenvironment{tablehere}
{\def\@captype{table}}
{}
\makeatother

\begin{document}
	%————起始————
	\vspace*{-5em}
	\begin{center}
		\includegraphics[width=0.6\textwidth]{Picture//USTC}\\
		\Large \textbf{大学物理-综合实验|数据处理}\\[5mm]

		\normalsize
		\begin{tabular}{ll}
			姓名 & \textbf{王元叙}\\
			学号 & \textbf{PB22000195}\\
			班级 & \textbf{22级少年班学院5班}\\
			日期 & \textbf{2023年10月09日}\\	
		\end{tabular}\\[5mm]

		\LARGE \textbf{\nexp}\\[5mm]	

	\end{center}
	%————————————

	%————正文————
	\section{数据分析}

	\subsection{测量 LM35 特性曲线}		
	\begin{tablehere}
		\caption*{\bf 表1 LM35 电压-温度特性曲线拟合图像}
		\noindent	
		\begin{center}
			\newcolumntype{Y}{>{\centering\arraybackslash}X}
			\begin{tabularx}{0.95\textwidth}{|Y|Y|Y|Y|Y|Y|Y|}
				\hline
				温度 & 30.2  & 40.0  & 50.1  & 60.1  & 70.0  & 80.2  \\ \hline
				电压 & 0.302 & 0.409 & 0.524 & 0.626 & 0.721 & 0.824 \\ \hline
			\end{tabularx}
			\vspace*{1em}
		\end{center}
	\end{tablehere}
	
	
	\begin{figurehere}
		\centering
		\includegraphics[width=0.85\linewidth]{pics/1.png}
		\caption*{\bf 图1: LM35 特性曲线线性拟合作图}
	\end{figurehere}

	通过软件计算可以得到,LM35传感器的灵敏度为
	$$
	\begin{aligned}
		K &= \frac{\Delta U}{\Delta T} \\
		&= 0.01042 \mathrm V/ ^\circ \mathrm C\\
		&= 10.42 \mathrm mV/ ^\circ \mathrm C
	\end{aligned}
	$$



	\subsection{使用 LM35 和放大电路组装温度计}

	使用标准温度计对控温仪作 37.0 $^\circ \mathrm C$ 的校正,并在 37.0 $^\circ \mathrm C$ 的条件下调节放大电路的校正旋钮使得控温仪、标准温度计、组装温度计示数均保持一致。

	\begin{tablehere}
		\caption*{\bf 表2 保持样品电流不变测量霍尔系数数据}
		\noindent	
		\begin{center}
			\newcolumntype{Y}{>{\centering\arraybackslash}X}
			\begin{tabularx}{0.95\textwidth}{|Y|Y|Y|Y|Y|Y|Y|Y|Y|}
				\hline
				控温仪 & 35.0 & 36.0 & 37.0 & 38.0 & 39.0 & 40.0 & 41.0 & 42.0 \\ \hline
				组装  & 35.0 & 35.9 & 37.0 & 38.1 & 39.1 & 40.0 & 41.2 & 42.3 \\ \hline
				标准  & 34.8 & 35.8 & 37.0 & 38.3 & 39.3 & 40.2 & 41.4 & 42.6 \\ \hline
				差异  & 0.2  & 0.1  & 0.0  & 0.2  & 0.2  & 0.2  & 0.2  & 0.3  \\ \hline
			\end{tabularx}
			\vspace*{1em}
		\end{center}
	\end{tablehere}

	\begin{figurehere}
		\centering
		\includegraphics[width=0.85\linewidth]{pics/2.png}
		\caption*{\bf 图2: LM35 特性曲线线性拟合作图}
	\end{figurehere}

	计算可以得到该组装温度计的线性度为:$\frac{0.3 ^\circ \mathrm C}{42.3 ^\circ \mathrm C} = 0.709 \%$

	用校准过的组装温度计测量手心的温度,得到的结果为:36.1 $^\circ \mathrm C$

	用校准过的组装温度计测量眉心的温度,得到的结果为:36.9 $^\circ \mathrm C$
\end{document}