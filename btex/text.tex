
给定函数 $f\colon \Omega\rightarrow \R$,我们上次课定义它的微分 $df(x)$。当 $x$ 给定的时候,这是一个 $\R^n$ 上的线性函数。
仿照一维的理论,对一阶导数求导就应该得到二阶导数。
然而,此时为了定义 $f$ 的二阶导数,我们就需要对 $x\mapsto df(x)$ 求微分。
此时,(不严格地讲)映射 $f\colon x\mapsto df(x)$ 是一个从 $\Omega$ 到 ${\rm Hom}(\R^n,\R)\simeq \R^n$ 的映射,这是向量值的函数。
所以,我们自然地想到对向量值的函数 $f\colon \Omega \rightarrow \R^m$ 定义微分。
当然,我们还可以考虑 $df(x)$ 的每个分量(依赖于坐标系的选取)然后要求每个分量都可微,至少从形式上看,后一种做法失于简洁和美观。

\begin{defn}[微分]
假设 $\Omega$ 是 $\R^n$ 中的区域,$\Omega'$ 是 $\R^m$ 中的区域,给定映射 $f\colon\Omega\rightarrow \Omega'$。如果存在线性映射
\[df\big|_{x=x_0}=df(x_0)\colon\mathbb{R}^n\rightarrow \mathbb{R}^m,\]
使得对于 $\R^n$ 中的 $v\rightarrow 0$ 时,我们有
\[f(x_0+v)=f(x_0)+df(x_0)v+o(v),\] 
即
\[\lim_{v\rightarrow 0}\frac{|f(x_0+v)-f(x_0)-df(x_0)v|}{|v|}=0,\]
我们就称 $f$ 在 $x_0$ 处可微并且称线性映射 $df(x_0)$ 是 $f$ 在 $x_0$ 的{\bf 微分}。
如果 $f$ 在 $\Omega$ 的每个点处都可微,我们就称 $f$ 是 $\Omega$ 上面的{\bf 可微映射}。
\end{defn}

\begin{rem}
\begin{enumerate}[label = \arabic*)]
\item 在上述微分的定义中,我们完全没有用到 $\R^n$ 和 $\R^m$ 上的坐标系。实际上,在下面的极限中
\[\lim_{v\rightarrow 0}\frac{{\color{blue}|}\overbrace{f(\underbrace{x_0+v}_{\text{用到 $\R^n$ 上的加法结构}})-f(x_0)-df(x_0)v}^{\text{用到 $\R^m$ 上的加法结构}}{\color{blue}|}}{{\color{red}|}v{\color{red}|}}=0,\]
我们只用到了 $\R^n$ 和 $\R^m$ 上的线性结构和它们上面的范数(我们分别用了蓝色和红色表示,其中红色的是定义域 $\R^n$ 上的范数,蓝色的是值域 $\R^m$ 上的范数)。

据此,我们可以将上述定义进行推广:给定赋范线性空间 $(V_1,\|\cdot\|_1)$ 和 $(V_2,\|\cdot\|_2)$,
$\Omega_1 \subset V_1$ 和 $\Omega_2 \subset V_2$ 是非空的开集,$f\colon \Omega_1\rightarrow \Omega_2$是映射。如果存在线性映射
\[df\big|_{x=x_0}=df(x_0)\colon V_1\rightarrow V_2,\]
使得对于 $V_1$ 中的 $v\rightarrow 0$ 时,我们有
\[\lim_{v\rightarrow 0}\frac{\|f(x_0+v)-f(x_0)-df(x_0)v\|_2}{\|v\|_1}=0,\]
我们就称 $f$ 在 $x_0$ 处可微并且称线性映射 $df(x_0)$ 是 $f$ 在 $x_0$ 的{\bf 微分}。


\item 假设$f$在$\Omega$上可微。那么,给定$x\in \Omega$, $df(x) \in {\rm Hom}(\R^n,\R^m)\approx \R^{mn}$(可以视作是$m\times n$的矩阵,这里我们用坐标比较方便)也在一个向量空间中取值的。
  (然而,如果在一般的(无限维的)的赋范线性空间上定义微分,我们会要求 $df(x) \in {\rm Hom}(V,W)$ 是所谓的连续线性映射,这里不展开讨论,有兴趣的同学可以在泛函分析的课程上学习)。
  所以,当 $x$ 变化的时候,我们就得到一个映射
\[\Omega \rightarrow \R^{nm},  \ \ x\mapsto df(x).\]
我们可以对它求导数来定义它的微分。高阶的微分不是这门课程的重点。

\item 假设 $f$ 在 $x\in \Omega$ 处可微分,我们就有映射
\[df(x_0)\colon\mathbb{R}^n\rightarrow \mathbb{R}^m.\]
由于这些映射依赖于点,特别地,依赖于 $x_0\in \Omega$(和 $f(x_0)\in \Omega'$),我们用 $T_{x_0}\Omega$ 代表它的定义域的线性空间 $\R^n$,
用 $T_{f(x_0)}\Omega'$ 代表它的定义域的线性空间 $\R^m$,这样子,我们形式上就有
\[df(x_0)\colon T_{x_0} \Omega \rightarrow T_{f(x_0)}\Omega'.\]
符号 $T_{x_0}\Omega$ 代表的是 $\Omega$ 在 $x_0$ 处的切空间(=切平面),$T_{f(x_0)}\Omega'$ 代表的是 $\Omega'$ 在 $f(x_0)$ 处的切空间,
我们会有专门的例子来理解这个对象,目前大家可以暂时将它们理解为好的记号。
\end{enumerate}
\end{rem}


我们上次课定义了方向导数和偏导数,这都是一维的对象。下面的命题表明,我们可以用偏导数这些一维的对象来描述 $df(x_0)$ 这个高维的对象:
\begin{prop}[微分的计算]
假设 $V=\R^n$ 和 $W=\R^m$,我们在 $V$ 上用坐标系 $\{x_i\}_{i=1,\cdots,n}$,
在 $W$ 上用坐标系 $\{y_j\}_{j=1,\cdots,m}$(把空间写成 $V$ 和 $W$ 是强调这些空间可以不用具体的坐标来描述)。
考虑$f\colon V\rightarrow W$(我们也可以考虑 $f$ 定义在 $V$ 中某个区域上),用坐标来写,我们有:
\[x\mapsto f(x)=\bigl(f_1(x_1,\cdots,x_n),f_2(x_1,\cdots,x_n),\cdots, f_m(x_1,\cdots,x_n)\bigr).\]
有时候还写成
\[y_1=f_1(x_1,\cdots,x_n), \ y_2=f_2(x_1,\cdots,x_n),\cdots, y_m=f_m(x_1,\cdots,x_n).\]
那么,我们有

\begin{enumerate}[label = \arabic*)]
\item 假设 $f$ 在 $x_0$ 处可微,那么每个分量函数 $f_j$ 在 $x_0$ 处都可微,其中 $j=1,2,\cdots, m$。
\item 如果每个分量函数 $f_j$ 在 $x_0$ 处都可微(其中 $j=1,2,\cdots, m$),那么 $f$ 在 $x_0$处可微。
\end{enumerate}
特别地,如果 $f$ 在 $x_0$ 处可微,那么映射 $df(x_0)\colon\R^n\rightarrow \R^m$ 可以用 $m\times n$ 的矩阵
\[\Bigl(\frac{\partial f_i}{\partial x_j}(x_0)\Bigr)_{i=1,\cdots,n \atop j=1,\cdots,m}\]
来表示(我们将这个矩阵称作是 $f$ 在 $x$ 处的{\bf Jacobi 矩阵},并记作 ${\rm Jac}(f)$ 或者 $\mathbf{J}(f)$,{\color{blue}它只是微分在一个特殊的坐标系下的表达})。
\end{prop}

\begin{proof}
我们首先证明,$f$ 在 $x_0$ 可微等价于每个分量 $f_j$($j=1,\cdots,m$)都可微。
假设 $f$ 在 $x_0$ 处可微,此时 $df\colon\R^n\rightarrow \R^m$ 有定义并且是线性映射。
由于我们在 $\R^n$ 上选定了基 $\{\frac{\partial}{\partial x_i}\}_{i\leqslant n}$,在 $\R^m$ 上选定了基 $\{\frac{\partial}{\partial y_j}\}_{j\leqslant m}$,
我们可以把这个线性映射用矩阵 $\bigl(J_{ij}\bigr)_{i\leqslant n, \atop j\leqslant m}$ 来表示。

首先,用分量表达,我们有
\begin{align*}
\frac{|f(x_0+v)-f(x_0)-df(x_0)v|}{|v|}&=\frac{\bigl|\bigl(\cdots, f_j(x_0+v)-f_j(x_0),\cdots\bigr)-\bigl(\cdots, \sum_{i=1}^n J_{ji} v_i ,\cdots\bigr)\bigr|}{|v|}\\
&=\frac{\sqrt{\sum\limits_{j=1}^n \bigl|f_j(x_0+v)-f_j(x_0)-\sum\limits_{i=1}^n J_{ji} v_i \bigr|^2 }}{|v|}.
\end{align*}

由于当 $v\rightarrow 0$ 时,上述左边为 $o(1)$,所以,限制到每个分量,我们就有
\[o(1)\geqslant \frac{ \bigl|f_j(x_0+v)-f_j(x_0)- \sum\limits_{i=1}^n J_{ji} v_i \bigr| }{|v|}. \]
按定义,这表明 $f_j$ 是可微分的(因为我们用线性映射在 $x_0$ 附近逼近了 $f_j$)。
反过来,假设对每个 $j\leqslant m$,我们都有
 \[\frac{ \bigl|f_j(x_0+v)-f_j(x_0)-\sum\limits_{i=1}^n J_{ji} v_i \bigr| }{|v|}=o(1),\]
 那么,
\begin{align*}
\frac{|f(x_0+v)-f(x_0)-df(x_0)v|}{|v|}&=\frac{\sqrt{\sum\limits_{j=1}^n  \bigl|f_j(x_0+v)-f_j(x_0)-\sum\limits_{i=1}^n J_{ji} v_i \bigr|^2 }}{|v|}\\
&\leqslant \sum_{j=1}^n \frac{\bigl|f_j(x_0+v)-f_j(x_0)-\sum\limits_{i=1}^n J_{ji} v_i \bigr| }{|v|}=n\times o(1)=o(1),
\end{align*}

所以 $df(x_0)$ 存在。

我们令 $v=t\frac{\partial}{\partial x_{i_0}}$,即 $v_{i_0}=t$ 而其它分量 $=0$。此时,根据微分的定义,上面的式子的左边是 $o(1)$ 项($t\rightarrow 0$)。计算右边,我们得到
\[
o(1)=\frac{\sqrt{\sum\limits_{j=1}^n \Bigl|f_j\bigl(x_0+\overbrace{(0,\cdots,0,t,0\cdots,0)}^{\text{只有第$i_0$个位置非$0$}}\bigr)-f_j\bigl(x_0\bigr)- J_{ji_0} t \Bigr|^2 }}{t}.
\]
对于一个特定的指标 $j_0$,我们自然有
\begin{align*}
&\sqrt{\sum\limits_{j=1}^n  \Bigl|f_j\bigl(x_0+ (0,\cdots,0,t,0\cdots,0)\bigr)-f_j\bigl(x_0\bigr)- J_{ji_0} t \Bigr|^2 }\\
\geqslant&\Bigr| f_{j_0}\bigl(x_0+ (0,\cdots,0,t,0\cdots,0)\bigr)-f_{j_0}\bigl(x_0\bigr)- J_{j_0i_0} t \Bigr|.
\end{align*}
所以,
\[o(1)={\Bigl| f_{j_0}\bigl(x_0+ (0,\cdots,0,t,0\cdots,0)\bigr)-f_{j_0}\bigl(x_0\bigr)- J_{j_0i_0} t \Bigr|}{t}.\]
按照定义,这表明 $f_{j_0}$ 的沿着 $x_{i_0}$ 偏导数存在并且等于 $J_{j_0i_0}$,这表明
\[J_{ji}=\frac{\partial f_j}{\partial x_i}(x_0).\]
命题得证。\end{proof}

\begin{rem}
上述命题表明,映射可求微分等价于其分量可求微分,所以,我们可以通过继续对分量求微分来引入 $k$-次可导的概念(就是每次求完微分之后这个微分的每个分量都能再求微分)。
所以,我们可以定义 $C^k(\Omega,\mathbb{R}^m)$,这是 $k$ 次微分仍然连续的映射的空间。
根据上次课程用偏导数判定微分存在性的定理,我们知道只要 $f$ 的连续 $k$ 次偏导数(可能是沿着不同方向的)存在并且{\color{red}连续},那么映射就是 $C^k$ 的。
这是一个非常方便有效的判断方式。
\end{rem}

我们现在研究符合映射的微分,也就是所谓的链式法则。
\begin{prop}[链式法则]
假设 $\Omega_j \subset \R^{m_j}$ 是开集,其中 $j=1,2,3$,$f\colon\Omega_1\rightarrow \Omega_2$,$g\colon\Omega_2\rightarrow \Omega_3$是映射。
假设 $f$ 在点 $x_1\in \Omega_1$ 处可微,$g$ 在点 $x_2=f(x_1)\in \Omega_2$ 处可微,那么复合映射 $g \circ f$ 在 $x_0$ 处可微,并且
\[\bigl(d (g\circ f) \bigr)(x_0)=(dg)(f(x_0))\circ df(x_0).\]
\end{prop}

\begin{rem}
上述映射的复合可以用下面的交换图来表示:
\[\begin{tikzcd}
  \Omega_1 \arrow[r,"f"] \arrow[dr, "g\circ f"']
    & \Omega_2 \arrow[d,"g"]\\
&\Omega_3 \end{tikzcd}\]
那么,它们所对应的微分(在线性的层次上)也可以用类似的交换图来表示:
\[\begin{tikzcd}
  \R^{m_1} \arrow[r,"df(x_1)"] \arrow[dr, "d(g\circ f)(x_1)"']
    &   \R^{m_2} \arrow[d,"dg(f(x_1))"]\\
&  \R^{m_3}\end{tikzcd}\]
我们之前引入的符号更好的描述了这个场景:对于映射 $df(x_1)\colon \R^{m_1}\rightarrow \R^{m_2}$,
我们将 $x_1$ 所对应的 $\R^{m_1}$ 记作 $T_{x_1}\Omega_1$,将 $\R^{m_2}$ 记作是 $T_{f(x_1)}\Omega_2$,那么,我们有映射
\[df\bigl|_{x=x_1}\colon T_{x_1}\Omega_1\rightarrow T_{f(x_1)}\Omega_2.\]
从而,上面的交换图表可以写成
\[\begin{tikzcd}
  T_{x_1}\Omega_1 \arrow[r,"df|_{x=x_1}"] \arrow[dr, "d(g\circ f)|_{x=x_1}"']
    &   \ T_{f(x_1)}\Omega_2 \arrow[d,"dg|_{x=f(x_1)}"]\\
&   T_{g(f(x_1))}\Omega_3 \end{tikzcd}\]
\end{rem}

\begin{proof}
链式法则的推导与一维的情形如出一辙:令 $x_2=f(x_1)\in \Omega_2$,按照定义有
\begin{align*}
&f(x_1+h)=f(x_1)+df(x_1)h+\delta(h), \\ 
&g\bigl(f(x_1)+\ell\bigr)=g\bigl(f(x_1)\bigr)+dg(x_2)\ell +\Delta(\ell),
\end{align*}
其中 $h\in \R^{m_1}$,$\ell\in \R^{m_2}$,$\lim\limits_{h \rightarrow 0}\frac{|\delta(h)|}{|h|}=\lim_{\ell \rightarrow 0}\frac{|\Delta(\ell)|}{|\ell|}=0$。据此,我们有
\begin{align*}
g(f(x_1+h))-g(f(x_1))&=g\bigl(f(x_1)+df(x_1)h+\delta(h)\bigr)-g(f(x_1))\\
&=df(x_2)\bigl(df(x_1)h+\delta(h)\bigr)+\Delta(f'(x_0)h+\delta(h))\\
&=\underbrace{df(x_2)\bigl(df(x_1)h\bigr)}_{=dg(x_2)\circ df(x_1)(h)}+df(x_2)\bigl(\delta(h)\bigr)+\Delta(f'(x_0)h+\delta(h)).
\end{align*}
所以,
\begin{align*}
&\frac{g(f(x_1+h))-g(f(x_1))-dg(x_2)\circ df(x_1)(h)}{h}\\
=&\frac{df(x_2)\bigl(\delta(h)\bigr)}{h}+\frac{\Delta(f'(x_0)h+\delta(h))}{h}\\
\leqslant & C \Bigl|\frac{\delta(h)}{h}\Bigr|+\underbrace{\Bigl| \frac{\Delta(f'(x_0)h+\delta(h))}{|f'(x_0)h+\delta(h)|}\Bigr|}_{o(1)}+\underbrace{\Bigl|\frac{|f'(x_0)h+\delta(h)|}{h} \Bigr|}_{\leqslant C_1}.
\end{align*}
由此可见,这是一个 $o(1)$ 项,按照微分的定义,
\[d(g\circ f)(x_1) = dg(x_2) \circ df(x_1).\]
这就完成了证明。
\end{proof}

作为推论,我们可以计算反函数(逆映射)的微分:

\begin{coro}
给定区域 $\Omega_1\subset \R^{n_1}$ 和 $\Omega_2\subset \R^{n_2}$ 和可微映射 $f\colon \Omega_1 \rightarrow \Omega_2$。
假设 $f$ 是双射并且其逆映射 $f^{-1}\colon\Omega_2\rightarrow \Omega_1$ 是可微的,那么
\begin{itemize}
\item $n_1=n_2$;
\item $df(x)$是可逆的(等价于 ${\rm Jac}(f)(x)$ 的行列式是非零的)。
\end{itemize}

此时,对于任意的 $y\in \Omega_2$,我们有
\[df^{-1}(y)=\Bigl(df\big |_{x=f^{-1}(y)}\Bigr)^{-1}.\]
\end{coro}
\begin{proof}
我们令 $\Omega_3=\Omega_1$,$g=f^{-1}$,$x_1=x$,$x_2=y$,$g\circ f ={\rm Id}$,其中
\[{\rm Id}\colon{\Omega_1}\rightarrow \Omega_1,  \ \ x\mapsto x,\]
是单位映射,它的微分在每个点处都是单位映射(线性)。根据链式法则,我们就有
\[{\rm Id}= dg(y) \circ df(x).\]
根据矩阵的秩的理论,我们知道 $n_1\leqslant n_2$。用 $f^{-1}$ 替换 $f$,我们就得到 $n_2\leqslant n_1$。这就证明了维数的部分。上面的等式已经蕴含了逆映射的微分的计算。
\end{proof}

\begin{exam}[指数映射的微分]
上个学期我们对于 $n\times n$ 的矩阵定义了指数映射
\[\exp\colon\mathbf{M}_n(\R) \rightarrow\mathbf{M}_n(\R), \ A\mapsto e^A=\sum_{k=0}^\infty\frac{A^k}{k!}.\]
我们现在计算它的微分 $d\exp$。固定 $A \in \mathbf{M}_n(\R)$,我们要找到
\[d\exp(A)\colon \mathbf{M}_n(\R) \rightarrow \mathbf{M}_n(\R),\]
其中,我们把 $\mathbf{M}_n(\R)$ 视作是 $\R^{n^2}$。对于任意较小的 $V\in \mathbf{M}_n(\R)$,我们有
\begin{align*}
e^{A+V}-e^V&=\sum_{n=0}^\infty \frac{1}{n!}\left((A+V)^n-A^n\right)
\end{align*}
现在强行展开 $(A+V)^n-A^n$(注意矩阵 $A$ 和 $V$ 的乘法未必交换)。通过将 $V$ 的二次项(以及更高次数的项)放到一起,我们得到
\[(A+V)^n-A^n=\sum_{k=0}^n A^k V A^{n-1-k}+Q_n(V).\]
二项式展开的一共不超过 $2^n$ 项,所以 $Q_n(V)$ 中至多有 $2^n$ 项。
我们上学期证明过(无论你选取什么样的范数),存在常数 $c$(依赖于范数),使得对任意的 $n\times n$ 的矩阵 $A$ 和 $B$,我们都有
\[\|A\cdot B\|\leqslant c \|A\|\|B\|.\]
上述 $Q_n(V)$ 的一个通项形如 $A AV VAA\cdots AA$,这是一个由 $n$ 个 $A$ 和 $V$ 排出来的长度为 $n$ 的字符串,其中至少有 $2$ 个 $V$。
我们可以要求 $\|V\|\leqslant \|A\|$,因为最终我们会令 $V\rightarrow 0$(除非 $A=0$,此时 $Q_n(V)=V^n$,下面的结论仍然成立),所以
\[\|A AV VAA\cdots AA\|\leqslant c^n \|A\|\|A\|\|V\|\|V\|\|A\|\|A\|\cdots \|A\|\|A\|.\]
那么,我们得到
\[\|Q_v(V)\|\leqslant 2^n \times \bigl(c^n \|V\|^2\|A\|^{n-2}\bigr).\]
从而,我们有
\begin{align*}
&\Bigl\|\exp(A+V)-\exp(A)-\sum_{n=0}^\infty \frac{1}{n!}\Bigl(\sum_{k=0}^n A^k V A^{n-1-k}\Bigr)\Bigr\|\\
\leqslant&\sum_{n=0}^\infty \Bigl\|\frac{1}{n!}Q_n(V)\Bigr\|\leqslant  \left(\sum_{n=0}^\infty\frac{(2c\|A\|)^n}{n!}\right)\|V\|^2=e^{2c\|A\|}\|V\|^2.
\end{align*}
那么,我们注意到右端的项是 $o(\|V\|)$ 并且 $\sum\limits_{n=0}^\infty \frac{1}{n!}\Bigl(\sum_{k=0}^n A^k V A^{n-1-k}\Bigr)$ 是收敛的。所以,
\[d\exp(A)(V)=\sum\limits_{n=0}^\infty \frac{1}{n!}\Bigl(\sum_{k=0}^n A^k V A^{n-1-k}\Bigr)\]
特别地,如果 $A$ 和 $V$ 可交换,那么 $d\exp(A)(V)=\exp(A)V$。我们还有
\[d\exp(0)={\rm Id}.\]
\end{exam}

有了链式法则,我们可以讨论更换坐标系的问题。这是个核心的话题,我们在中学的时候就已经在使用这个概念,比如说我们经常在极坐标和Descartes坐标系之间转换。我们首先用映射的语言来描述极坐标:令
\[\Omega_1 = \R^2-\bigl\{(x,0)\bigm| x\geqslant 0\bigr\} \subset \R^2,  \ \ \Omega_2=\R_{>0}\times (0,2\pi)=\bigl\{(r,\vartheta)\bigm| r>0, \vartheta \in (0,2\pi)\bigr\}.\]

{{image|MathAnalysis-N0201.svg|center}}

我们通常用的 $x=r\cos\vartheta$ 和 $y=r\sin \vartheta$ 可以用如下的映射来写:
\[\Phi\colon \Omega_2\rightarrow \Omega_1, \ \ (r,\vartheta)\mapsto (r\cos\vartheta, r\sin \vartheta). \]
由于在 $\Omega_1$ 上我们给定了 $(x,y)$ 作为坐标,在 $\Omega_2$ 上我们给定了 $(r,\vartheta)$ 作为坐标,所以我们可以用 Jacobi 矩阵来表示上述映射的微分:
\[d\Phi = {\rm Jac}(\Phi) = \begin{pmatrix}
   \frac{\partial x}{\partial r}&  \frac{\partial x}{\partial \vartheta} \\
    \frac{\partial y}{\partial r} &  \frac{\partial y}{\partial \vartheta} \\
  \end{pmatrix} = \begin{pmatrix}
  \cos\vartheta&    -r\sin\vartheta \\
     \sin\vartheta &    r\cos\vartheta\\
  \end{pmatrix}.\]
这个线性映射自然是可逆的,它的行列式是 $r$。
