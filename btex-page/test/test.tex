\section{基本习题}
\subsection{习题A(Stieltjes积分)}
如果不加说明,$\mu$ 表示的是有界闭区间 $I=[a,b]$ 上的一个递增的函数。下面题目中A1),A3),A4),A5),A6)是对概念的直接验证,可以不提交作业。
\begin{enumerate}[label=A\arabic*)]
\item 对于任意两个分划 $\sigma,\sigma' \in \mathcal{S}(I)$,我们有
\[\underline{S}_\mu(f;\sigma)\leqslant \overline{S}_\mu(f;\sigma').\]
\item 对任意的 $\rho\in C\bigl([a,b]\bigr)$,$\rho \geqslant 0$,$\mu(x)=\displaystyle\int_a^x \rho(x)dx$。证明,对任意的 $f\in \mathcal{R}\bigl([a,b]\bigr)$,都有 $f\in \mathcal{R}\bigl([a,b];\mu\bigr)$ 并且
\[\int_a^b fd\mu= \int_a^b f(x)\rho(x)dx.\]
特别地,假设 $\mu$ 连续可微($\mu$ 默认是递增的),我们有
\[\int_a^b f d\mu = \int_a^b f \mu'.\]
\item 证明,$\mathcal{R}(I,\mu)$ 是 $\R$-线性空间并且积分算子
\[\displaystyle \int_{a}^b\cdot \,d\mu\colon  \mathcal{R}(I;\mu)\rightarrow \R,\ \  f \mapsto \int_a^b fd\mu\]
是线性映射。
\item 假设 $f_1,f_2\in \mathcal{R}(I,\mu)$。如果对任意的 $x\in I$,我们都有 $f_1(x) \leqslant f_2(x)$,那么,
\[\int_a^b f_1d\mu \leqslant \int_a^b f_2d\mu.\]
\item (区间可加性)如果 $f\in \mathcal{R}([a,b];\mu)$,那么对任意的 $c\in [a,b]$,$f$ 在 $[a,c]$ 和 $[c,b]$ 上的限制都是Stieltjes可积的并且
\[\int_a^b f d\mu=\int_a^c f d\mu+\int_c^b fd\mu.\]
\item 如果 $f,g\in  \mathcal{R}([a,b];\mu)$,那么 $f\cdot g\in  \mathcal{R}([a,b];\mu)$。
\item 我们在区间 $[0,\infty)$ 定义Stieltjes积分:假设 $f\in C([0,\infty)$ 是有界连续函数,定义
\[\int_0^\infty f d\mu= \lim_{M\rightarrow \infty}\int_0^M f d\mu.\]
假设 $\{\alpha_n\}_{n\geqslant 1}$ 是正实数所组成的序列并且 $\displaystyle \sum_{n=1}^\infty \alpha_n$ 收敛,定义递增函数 $\mu =\displaystyle\sum_{n=1}^\infty \alpha_n \mathbf{1}_{\geqslant n}$。
那么,我们有
\[\int_1^\infty fd\mu = \sum_{n=1}^\infty \alpha_n f(n).\]
\item (第一积分中值定理)$f, g\in \mathcal{R}([a,b];\mu)$ 是实值Riemann可积函数。我们假设对任意的 $x\in [a,b]$,$g(x)\geqslant 0$。令
\[m=\displaystyle \inf_{x\in I} f(x), \ \ M=\displaystyle \sup_{x\in I} f(x).\]
那么,存在 $\ell \in [m,M]$,使得
\[\int_a^b fgd\mu= \ell \int_{a}^b gd\mu.\]
特别地,如果进一步要求 $f$ 是连续函数,那么存在 $\xi \in [a,b]$,使得
\[\displaystyle \int_a^b fgd\mu= f(\xi) \int_{a}^b gd\mu.\]
\item 你是否可以构造一个Stieltjes积分来说明Abel求和法也是Stieltjes积分意义下的分部积分公式的特例?
\end{enumerate}

\subsection{习题B(反常积分收敛的判断)}
以下记号中,$b$ 可以是 $\infty$。
\begin{enumerate}[label=B\arabic*)]
		\item (Cauchy判别法) 假设 $f\colon  [a,b) \to \mathbb{R}$,对任意 $b^- < b$,$f$ 在 $[a,b^-]$ 上可积。证明,反常积分 $\displaystyle \int_a^b f(x)dx$ 存在的充要条件是: 对任意 $\varepsilon > 0$, 存在 $b(\varepsilon)  \in (a,b)$, 使得对任意 $b', b'' > b(\varepsilon)$, $\displaystyle		\left|\int_{b'}^{b''} f(x)dx\right| < \varepsilon$。

		\item (比较判别法,已经证明,可不交作业) 如果 $|f|$ 在 $[a,b)$ 上反常可积,就称积分 $\displaystyle \int_a^b f(x)dx$ {\bf 绝对收敛}。证明,如果 $|f(x)| \leqslant F(x), x \in [a,b)$ 并且 $\displaystyle\int_a^b F(x)dx$ 收敛, 那么积分 $\displaystyle\int_a^b |f(x)|dx$ 收敛。
				\item 证明{\bf ~Dirichlet判别法}:假设 $f,g\colon [a,\infty)\rightarrow \R$ 满足
				\begin{itemize}
				\item $f$ 是连续函数并且存在 $A>0$,使得对任意的 $M \geqslant a$,我们都有
				\[\Bigl|\int_a^M f(x)dx\Bigr|\leqslant A.\]
				\item $g$ 是单调函数并且 $\displaystyle \lim_{x\rightarrow \infty}g(x)=0$。
				\end{itemize}
				那么反常积分 $\displaystyle\int_a^\infty f(x)g(x)dx$ 收敛。

			(你可以查阅文献来证明这个判别法,一个好的开始是先假设 $f$ 和 $g$ 都是连续可微的,一般的情形可以考虑Stieltjes积分。请对比级数收敛的 Dirichlet 判别法以及相应的证明,你可以看到Abel求和法和分部积分之间的相似性)

\item 证明{\bf ~Abel判别法}:假设 $f,g\colon [a,\infty)\rightarrow \R$ 满足
				\begin{itemize}
				\item 反常积分 $\displaystyle \int_a^\infty f(x)dx$ 存在。
				\item $g$ 是单调函数并且 $g$ 有界。
				\end{itemize}
				那么反常积分 $\displaystyle\int_a^\infty f(x)g(x)dx$ 收敛。
(请对比级数收敛的 Abel判别法以及相应的证明)

	       \item 判断下列积分的收敛性 (绝对收敛、条件收敛、发散)
	\begin{align*}
	& (1) \int_{0}^{+\infty} \frac{\log (1+x)}{x^p} dx
	&& (2)\int_{1}^{+\infty} \frac{\sin x}{x^p} dx && (3) \int_{1}^{+\infty} \sin x^2 dx\\
	& (4) \int_{0}^{+\infty} \frac{\sin^2 x}{x}dx  && (5) \int_{0}^{2\pi} \frac{dx}{\cos^p x \cos^q x}, ~ p,q >0 && (6) \int_{0}^{+\infty} x^p \sin (x^q) dx \\
	& (7) \int_{0}^{+\infty} \frac{x^p \sin x}{1 + x^q}, ~ q \geq 0 && (8) \int_{0}^{\pi/2} \frac{\log \sin x}{\sqrt{x}} dx && (9) \int_{e}^{+\infty} \frac{\log \log x}{\log x} \sin x dx
	\end{align*}
	\end{enumerate}

\section{题目C(振荡积分)(8字班的考试题之一,不交作业,鼓励讨论)}

我们采取下面的约定:$F(t)$ 和 $G(t)$ 是 $[1,+\infty)$ 上定义的两个函数,$\displaystyle  \lim_{t\rightarrow \infty}G(t)=0$。我们假设对任意的 $t\geqslant 1$,$G(t)\neq 0$。如果
\[\lim_{t\rightarrow \infty}\frac{F(t)}{G(t)}=1,\]
我们就说 $F(t)$ 和 $G(t)$ 是{\bf 均阶的}(即收敛到 $0$ 的速度是一样的)并记作 $F(t)\sim G(t)$。

\subsection{第一部分:相函数是实值的情形}

\begin{enumerate}[label=C\arabic*)]
\item $d>0$ 是给定的实数。假设 $g\in C^1([0,d])$。证明,存在常数 $C$(可能依赖于 $g$ 和 $d$),使得
\[\Bigl|\int_0^d e^{-tx}g(x)dx\Bigr| \leqslant \frac{C}{t}.\]

\item $d>0$ 是给定的实数。假设 $g\in C([0,d])$ 并且 $g(0)\neq 0$。证明,
\[\int_0^d e^{-tx}g(x)dx \sim \frac{g(0)}{t} .\]
(提示:利用积分换元 $y=tx$)

\item $d>0$ 是给定的实数。假设 $g\in C([0,d])$ 并且 $g(0)\neq 0$。证明,
\[\int_0^d e^{-tx^2}g(x)dx \sim \frac{\sqrt{\pi}\cdot g(0)}{2\sqrt{t}} .\]
(提示:我们课堂上已经证明了 $\displaystyle \int_0^\infty e^{-x^2}dx=\dfrac{\sqrt{\pi}}{2}$)
\end{enumerate}

对于 $t\geqslant 1$,$f\in C([a,b])$ 和 $\varphi \in C([a,b])$,我们定义函数
\[F(t)=\int_{a}^b e^{-t\varphi(x)}f(x)dx,\]
我们的目标是研究当 $t\rightarrow +\infty$ 时,$F(t)$ 的行为。

\begin{enumerate}[label=C\arabic*)]
\setcounter{enum-{list-level}}{3}
\item 假设 $\varphi \in C^1([a,b])$,并且对任意 $x\in [a,b]$,$\varphi'(x)\neq 0$。为了简单起见,我们进一步假设 $\varphi'(x)> 0$。令 $d=\varphi(b)-\varphi(a)$。证明,映射
\[\Phi\colon [a,b]\rightarrow [0,d], \ \ x\mapsto \varphi(x)-\varphi(a),\]
在 $[a,b]$ 上连续可微的并且是双射。

\item 假设 $\varphi \in C^1([a,b])$,并且对任意 $x\in [a,b]$,$\varphi'(x)\neq 0$。证明,如果 $f(a)\neq 0$,当 $t\rightarrow +\infty$ 时,
\[F(t)\sim\frac{f(a)}{\varphi'(a)}\frac{e^{-t\varphi(a)}}{t}.\]
(提示:利用C4)中的函数做积分换元)

\item 假设 $\varphi \in C^2([a,b])$,$\varphi'(a)=0$,$\varphi''(a)>0$ 并且对任意 $x\in (a,b]$,$\varphi'(x)>0$。令 $d=\sqrt{\varphi(b)-\varphi(a)}$。证明,映射
\[\Psi\colon [a,b]\rightarrow [0,d], \ \ x\mapsto \sqrt{\varphi(x)-\varphi(a)},\]
在 $[a,b]$ 上连续可微的并且是双射。

特别地,试计算 $\Psi'(a)$。

(提示:考虑 $\varphi$ 的二阶的Taylor展开。)

\item 假设 $\varphi \in C^2([a,b])$,$\varphi'(a)=0$,$\varphi''(a)>0$ 并且对任意 $x\in (a,b]$,$\varphi'(x)>0$。证明,如果 $f(a)\neq 0$,当 $t\rightarrow +\infty$ 时,
\[F(t)\sim \frac{\sqrt{\pi} f(a)}{\sqrt{2\varphi''(a)}}\frac{e^{-t\varphi(a)}}{\sqrt{t}}.\]
(提示:利用C6)中的函数做积分换元)

\item 给定两个函数 $f\in C\bigl((0,+\infty)\bigr)$,$\varphi \in C^2\bigl((0,+\infty)\bigr)$。我们假设
\begin{itemize}
\item 存在唯一的 $a\in (0,\infty)$,使得 $\varphi'(a)=0$;
\item $\varphi''(a)>0$,$f(a)\neq 0$;
\item 反常积分 $\displaystyle \int_{0}^\infty e^{-\varphi(x)}|f(x)|dx$ 收敛。
\end{itemize}
证明,当 $t\rightarrow +\infty$ 时,函数 $\displaystyle G(t)=\int_{0}^\infty e^{-t\varphi(x)}f(x)dx$ 满足
\[G(t)\sim \frac{\sqrt{2\pi} f(a)}{\sqrt{\varphi''(a)}}\frac{e^{-t\varphi(a)}}{\sqrt{t}}.\]
\item 对于 $x>0$,定义函数
\[\Gamma(x)=\int_{0}^\infty t^{x-1}e^{-t}dt.\]
对于正整数 $n$,试计算 $\Gamma(n)$。
\item 证明Stirling公式:
\[n!\sim \sqrt{2\pi}\frac{n^{n+\frac{1}{2}}}{e^n}.\]
(提示:我们有 $\displaystyle \Gamma(n+1)=n^{n+1}\int_0^\infty e^{-n(x-\log x)}dx$。)
\end{enumerate}

\subsection{第二部分:振荡积分}

对于 $\lambda \geqslant 1$,$f\in C^\infty([a,b])$ 和 $\varphi \in C^\infty([a,b])$,我们定义函数
\[I(\lambda)=\int_{a}^b e^{i\lambda \varphi(x)}f(x)dx,\]
我们的目标是研究当 $\lambda \rightarrow +\infty$ 时,$I(\lambda)$ 的行为。

\begin{enumerate}[label=C\arabic*)]
\setcounter{enum-{list-level}}{10}
\item 假设对任意 $x\in [a,b]$,$\varphi'(x)\neq 0$。我们定义映射
\begin{align*}
L\colon C^\infty([a,b]) \rightarrow C^\infty([a,b]), 	\ \ h  &\mapsto \frac{1}{i\lambda \varphi'(x)}h'(x),\\
M\colon C^\infty([a,b]) \rightarrow C^\infty([a,b]), \ \ h &\mapsto -\Bigl(\frac{h}{i \varphi'}\Bigr)'(x),\\
\end{align*}
假设 $g, h\in C^\infty([a,b])$。证明,如果 $h$ 在 $a$ 和 $b$ 附近恒为 $0$(即存在 $c>0$,使得当 $x\in [a,a+c]\cup [b-c,b]$ 时,$h(x)=0$),那么
\[\int_a^b h \cdot Lg  =\frac{1}{\lambda}\int_{a}^b g \cdot Mh.\]
\item 假设对任意 $x\in [a,b]$,$\varphi'(x)\neq 0$ 并且 $f$ 在 $a$ 和 $b$ 附近恒为 $0$。证明,对任意的 $n\in \mathbb{Z}_{\geqslant 1}$,都存在{\bf 不依赖}于 $\lambda$ 的常数 $c_n$,使得
\[|I(\lambda)|\leqslant \frac{c_n}{\lambda^n}.\]
(提示:注意到 $L e^{i\lambda\varphi(x)}=e^{i\lambda\varphi(x)}$。)
\item 假设存在 $\delta>0$,使得对任意 $x\in [a,b]$,$|\varphi'(x)|\geqslant \delta$ 并且 $\varphi'(x)$ 是 $[a,b]$ 上的单调函数。证明,存在{\bf 不依赖}于 $\lambda$,$\varphi$,$a$ 和 $b$ 的常数 $C_1$,使得
\[\bigl|\int_a^b e^{i\lambda\cdot \varphi(x)}dx\bigr|\leqslant \frac{C_1}{\lambda \delta}\]

\item 假设对任意 $x\in [a,b]$,$|\varphi''(x)|\geqslant 1$。证明,存在唯一的 $c\in [a,b]$,使得
\[|\varphi'(c)|=\displaystyle \inf_{x \in [a,b]}|\varphi'(x)|.\]
进一步证明,对任意的 $x\in [a,b]$,我们的都有
\[|\varphi'(x)|\geqslant |x-c|.\]

\item[C15)] 假设对任意 $x\in [a,b]$,$|\varphi''(x)|\geqslant 1$。证明,存在{\bf 不依赖}于 $\lambda$,$\varphi$,$a$ 和 $b$ 的常数 $C_2$,使得
\[\bigl|\int_a^b e^{i\lambda\cdot \varphi(x)}dx\bigr|\leqslant \frac{C_2}{\sqrt{\lambda}}.\]

\item[C16)] 假设对任意 $x\in [a,b]$,$|\varphi''(x)|\geqslant 1$。证明,存在{\bf 不依赖}于 $\lambda$,$\varphi$,$f$,$a$ 和 $b$ 的常数 $C_2$,使得
\[\bigl|\int_a^b e^{i\lambda\cdot \varphi(x)}f(x)dx\bigr|\leqslant \frac{C_2}{\sqrt{\lambda}}\Bigl(|f(a)|+\int_{a}^b |f'(x)| dx\Bigr).\]

\end{enumerate}
